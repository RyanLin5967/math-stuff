\documentclass[12pt]{article}
\usepackage{amsmath, amssymb}
\usepackage{amsthm}


\setlength{\parindent}{24pt}

\begin{document}

\begin{enumerate}
    \item (a) is equivalent to (i) because they are their contrapositives, and
          equal to (g) and (f). (h) is the contrapositive of (b) so they are equal and
          they are also equal to (e). By definition, (c) and (d) are
          equivalent.

    \item If x, y are both even integers $(P)$, then the sum is even $(Q)$. The converse
          is if a number is even$(Q)$, then it must have been added by two even
          integers$(P)$, but $\frac{3}{2} + \frac{1}{2} = 2$, which is even, so the
          converse is false.

    \item

    \item They are not equivalent. Let $P(x,y)$ equal $x < y$. Notice that the statement
          ``$\forall x \exists y$'' is always true since there is always a number less
          than another, but the statement ``$\forall y \exists x$'' says all numbers are
          less than $y$, which is false since there is no number less than all numbers.

    \item
          \begin{enumerate}
              \item Since we are given that $x$ is an integer, then we can seperate the cases where
                    $x$ is odd and even. \newline Case $x$ is odd:

                    \begin{quote}
                        By definition, we can represent $x$ as $2n + 1$ for some integer $n$.
                        Substituting that into $x^2+5x$, we get:
                        \begin{center}$x^2+5x = (2n+1)^2 + 5(2n+1)$ \end{center}
                        and after simplifying and combining like terms we get:
                        \begin{center}$4n^2 + 14n + 6$ \end{center}
                        Letting $m = 2n^2 + 7n + 3$, we have:
                        \begin{center}$2(2n^2+7n+3) = 2m$\end{center}
                        which is even.
                    \end{quote}
                    Case $x$ is even:
                    \begin{quote}
                        By definition, we can represent $x$ as $2n$ for some integer $n$.
                        Substituting into $x^2 + 5x$, we get
                        \begin{center}
                            $x^2+5x = (2n)^2 + 5(2n) = 4n^2 + 10n$
                        \end{center}
                        Letting $m = 2n^2 + 5n$, we have
                        \begin{center}
                            $2(2n^2+5n) = 2m$
                        \end{center}
                        which is also even.
                    \end{quote}
                    Therefore, for all integers, $x^2+5x$ is even. \hfill $\blacksquare$
              \item By definition, we can represent an odd integer as $2n+1$ for some integer $n$.
                    Substituting into $x^3-3x^2+5x+7$, we have
                    \begin{center}
                        $x^3-3x^2+5x+7 = (2n+1)^3-3(2n+1)^2+5(2n+1)+7$
                    \end{center}
                    and after expanding and collecting like terms, we get
                    \begin{center}
                        $8n^3 + 4n + 10$.
                    \end{center}
                    Letting $m = 4n^3 + 2n + 5$, we have
                    \begin{center}
                        $2(4n^3+2n+5) = 2m$,
                    \end{center}
                    which is even. Hence if x is odd, $x^3-3x^2+5x+7$ is even. \hfill $\blacksquare$
              \item Note that the a statement and its contrapositive are equivalent so proving that
                    the contrapositive is true will be sufficient. Since we are given that $x$ is
                    an integer, the contrapositive of the statement is if $x$ is even, then
                    $x^2+2x+3$ is odd. Letting $x = 2n$, we have
                    \begin{center}
                        $x^2+2x+3 = (2n)^2+2(2n) + 3 = 4n^2 + 4n + 3$.
                    \end{center}
                    Letting $m = 2n^2+2n+1$, we get
                    \begin{center}
                        $4n^2 + 4n + 3 = 2(2n^2+2n+1) + 1 = 2m + 1$,
                    \end{center}
                    which is odd. \hfill $\blacksquare$
              \item Since $x$ is an integer, we can seperate the cases into when it's odd and even.
                    \newline Case $x$ is even:
                    \begin{quote}
                        Let $x = 2m$ for some integer m (so $x$ is even by definition).
                        Substituting into $n^2$, we have
                        \begin{center}
                            $n^2 = (2m)^2 = 4m^2$.
                        \end{center}
                        Let $k = m^2$. Then
                        \begin{center}
                            $4m^2 = 4k$,
                        \end{center}
                        which is divisible by $4$.
                    \end{quote}
                    Case $x$ is odd:
                    \begin{quote}
                        Let $x = 2m + 1$ for some integer m (so $x$ is odd by definition).
                        Substituting into $n^2+1$, we get
                        \begin{center}
                            $n^2+1 = (2m+1)^2-1 = 4m^2 + 4m$
                        \end{center}
                        Let $k = m^2 + m$. Then
                        $4m^2 + 4m = 4(m^2+m) = 4k$,
                        Which is also divisible by 4.
                    \end{quote}
                    Therefore, for all integers, either $n^2$ or $n^2-1$ is a multiple
                    of $4$. \hfill $\blacksquare$
          \end{enumerate}
    \item Since this is 4 consecutive numbers, then there will be at least 2 even
          numbers, and one of them is a multiple of 4, and there will be one a factor of
          3. This gives at minimum factors 2, 3, and 4. This means the largest $n$ such
          that $x(x+1)(x+2)(x+3)$ is a multiple of n is $2 \times 3 \times 4 = 24$.
          \hfill $\blacksquare$
    \item First, assume that $\sqrt{3}$ is rational, so then it can be represented as a
          fraction of two integers, $a$ and $b$: $\frac{a}{b}$, where $b$ is positive and
          as small as possible. Now, square both sides to get
          \begin{center}
              $3 = \frac{a^2}{b^2}$,
          \end{center}
          and rearange to get
          \begin{center}
              $3b^2 = a^2$.
          \end{center}
          Now, notice that since the left side is a multiple of 3, then the right side
          must be a multiple of 3 as well, so we can let $a = 3c$ for some integer $c$.
          So now we have
          \begin{center}
              $3b^2 = (3c)^2 = 9c^2$,
          \end{center}
          and dividing both sides by 3 gives $b^2 = 3c^2$. Rearanging and taking the
          square root of both sides gives
          \begin{center}
              $\sqrt{3} = \frac{b}{c}$
          \end{center}
          with $b>c>0$. This is a contradiction since we have just written $\sqrt{3}$
          as a fraction with a denominator smaller than $b$. This means that our
          initial assumption that $\sqrt{3}$ is rational must be false. Thus, $\sqrt{3}$
          is irrational. \hfill $\blacksquare$\newline
          Proof for $\sqrt[3]{4}$:
          First, assume that $\sqrt[3]{4}$ is rational, so then it can be represented as a
          fraction of two integers, $a$ and $b$: $\frac{a}{b}$, where $b$ is positive and 
          as small as possible. Now, cube both sides to get
\end{enumerate}
\end{document}