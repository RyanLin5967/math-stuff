\documentclass[12pt]{article}
\usepackage{amsmath, amssymb}

\begin{document}

\begin{enumerate}
  \item Which of the following statements are equivalent to which others?
  \begin{enumerate}
    \item If $P$ then $Q$.
    \item If $Q$ then $P$.
    \item $P$ only if $Q$.
    \item $Q$ only if $P$.
    \item $P$ if $Q$.
    \item Whenever $P$, $Q$.
    \item $P$ implies $Q$.
    \item Not $P$ implies not $Q$.
    \item Not $Q$ implies not $P$.
  \end{enumerate}

  \item Give an example of statements $P$ and $Q$, such that $P$ implies $Q$, 
  but $Q$ does not imply $P$. Given a statement ``$P$ implies $Q$,'' 
  we call the statement ``$Q$ implies $P$'' the \textit{converse}. 
  Statements and their converses are not the same!

  \item Given a statement ``If $P$ then $Q$,'' the statement ``If not $P$ then not $Q$'' 
  is called the \textit{inverse}. Explain why the inverse and converse are always equivalent: 
  if one of them is true, then so is the other; and if one of them is false, then so is the other.

  \item Suppose that $P(x,y)$ is a statement about two numbers $x$ and $y$. 
  Are the following necessarily equivalent: 
  $\forall x \, \exists y : P(x,y)$ and $\exists y \, \forall x : P(x,y)$? 
  Either prove they are equivalent for all statements $P$, 
  or else give an example of a statement $P$ such that one of them is true and the other is false.

  \item Prove the following statements:
  \begin{enumerate}
    \item For all integers $x$, $x^2 + 5x$ is even.
    \item If $x$ is odd, then $x^3 - 3x^2 + 5x + 7$ is even.
    \item Suppose $x$ is an integer. If $x^2 + 2x + 3$ is even, then $x$ is odd.
    \item For all integers $n$, either $n^2$ or $n^2 - 1$ is a multiple of 4.
  \end{enumerate}

  \item Find, with proof, the largest integer $n$ such that, for all integers $x$, 
  $x(x+1)(x+2)(x+3)$ is a multiple of $n$.

  \item Mimic the proof of the irrationality of $\sqrt{2}$ to prove that $\sqrt{3}$ 
  and $\sqrt[3]{4}$ are irrational. What goes wrong when you try to use the same 
  argument to prove that $\sqrt{9}$ is irrational?

  \item Prove that the sum of a rational number and an irrational number is irrational. 
  Prove that there exist two irrational numbers whose sum is rational.

  \item Prove that the product of a \textit{nonzero} rational number and an 
  irrational number is irrational. Prove that there exist two irrational numbers 
  whose product is rational.

  \item Prove that $\log_2(3)$ is irrational. (In case you haven't seen logarithms yet, 
  $\log_2(3)$ is the number $x$ such that $2^x = 3$.)

  \item Note that $\big(\sqrt{2}^{\sqrt{2}}\big)^{\sqrt{2}} = 2$. 
  Explain how to use this fact to prove that there exist two irrational numbers $x$ and $y$ 
  such that $x^y$ is rational.

  \item Prove that there are no integers $x$ and $y$ such that $x^2 - 3y^2 = 2$.
\end{enumerate}

\end{document}
