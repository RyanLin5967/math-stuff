\documentclass[12pt]{article}
\usepackage{amsmath, amssymb}
\usepackage{amsthm}
\begin{document}
\begin{enumerate}
    % Q1
    \item The mistake is that you assume there is ``overlap'' between
    the two sets mentioned, $H_1, \ldots ,H_{n+1}$ and $H_2, \ldots ,H_(n+1)$. 
    When $n = 1$, the two sets are just $H_1$ and $H_2$, meaning there is no overlap
    between the two sets, hence the argument is completely false after the assumption
    that there is overlap since we cannot go from $n = 1$ to $n = 2$.
    % Q2
    \item First, we claim that the sum of the first $n$ odd number is $n^2$. The
    base case is $n = 1$, and using $n^2$ gives $1^2 = 1$, so the base case is true.
    Next, we assume that it is true for $n$, so we need to prove it's true for 
    $n+1$. Since it's true for $n$, we have: 
    \begin{align*}
        1 + 3 + 5 + \cdots + (2n-1) + (2n+1) = n^2 + (2n+1).
    \end{align*}
    Completing the square on the right side gives \begin{align*}
        n^2 + (2n+1) = (n+1)^2,
    \end{align*}
    so by induction, the sum of the first $n$ odd numbers is $n^2$.
    % Q3
    \item First, we try the base case, $n = 1$. So we have \begin{align*}
        \frac{1^{2}(1+1)^{2}}{4} = \frac{4}{4} = 1,
    \end{align*}
    so the formula holds true for $n = 1$. Next, we assume that it is true
    for $n$, so we need to prove it's true for $n+1$. Since it's true for $n$, 
    we have: \begin{align*}
        1^3 + 2^3 +\cdots+n^3+(n+1)^3 = \frac{n^2(n+1)^2}{4} + (n+1)^3.
    \end{align*}
    Simplifying the right side gives \begin{align*}
        \frac{n^2(n+1)^2}{4} + (n+1)^3 = (n+1)^2(\frac{n^2+4n+4}{4}) = \frac{(n+1)^2(n+2)^2}{4},
    \end{align*}
    which is what we get if $n+1$ was plugged into the original equation, hence, by 
    induction, the sum of the first $n$ cubes is $\frac{n^2(n+1)^2}{4}$. Using 
    sigma notation, we can rewrite this sum as $\sum_{i=1}^{n} i^3 = \frac{n^2(n+1)^2}{4}$.
    % Q4
    \item To prove this, we can consider cases when $n$ is in the forms $5k, 5k+1, 
    5k +2, 5k+3,$ and $5k+4$, for some integer $k$, and we will use $n^5-n = n(n+1)(n-1)(n^2+1)$ to show that:
    \begin{align*}
        \text{Case } n &= 5k\text{: } \\ &=5k(5k+1)(5k-1)((5k)^2+1) \text{, which is divisible by 5} \\
        \text{Case } n &= 5k+1\text{: } \\ &=(5k+1)(5k+2)(5k)((5k+1)^2+1) \\
        &= 5(5k+1)(5k+2)(k)((5k+1)^2+1)\text{, which is divisible by 5}\\
        \text{Case } n &= 5k+2\text{: } \\ &=(5k+2)(5k+3)(5k+1)((5k+2)^2+1) \\
        &= 5(5k+2)(5k+3)(5k+1)(5k^2+4k+1)\text{, which is divisible by 5}\\
        \text{Case } n &= 5k+3\text{: } \\ &=(5k+3)(5k+4)(5k+2)((5k+3)^2+1) \\
        &= 5(5k+3)(5k+4)(5k+2)(5k^2+6k+2)\text{, which is divisible by 5}\\
        \text{Case } n &= 5k+4\text{: } \\ &=(5k+4)(5k+5)(5k+3)((5k+4)^2+1) \\
        &= 5(5k+4)(k+1)(5k+3)((5k+4)^2+1)\text{, which is divisible by 5}
    \end{align*}
    So, for all nonnegative integers, $n$ , $5$ divides into $n^5-n$. If $n$ is negative, 
    then you can factor out a negative from $n^5-n$:
    \begin{align*}
        (-n)^5-(-n) = -(n^5-n),
    \end{align*}
    and the cases we used previously would also hold true since it would just have 
    a negative sign in front. So for all negative integers, $5$ also divides into $
    n^5 - n$.
    % Q5
    \item The base case is $n = 0$, which gives \begin{align*}
        (1+x)^0 &\geq 1+(0)x \\
        1 &\geq 1,
    \end{align*}
    which is always true, so it holds for the base case. Assuming it is true for $n$,
    we need to show it is true for $n+1$, so:
    \begin{align*}
        (1+x)^{n+1} &\geq 1+(n+1)x\\
        (1+x)^n(1+x) &\geq (1+nx) + x,
    \end{align*}
        but since already assumed it is true for $n$, we are left with 
    \begin{align*}
        1+x \geq x,
    \end{align*}
    which is always true, so by induction, for $x > -1$, $(1+x)^n \geq 1 + nx$.
    % Q6
    \item First, we try the base cases, $n = 0$ and $n = 1$:
    \begin{align*}
        F_0 &= \frac{1}{\sqrt{5}} \left( \left( \frac{1 + \sqrt{5}}{2} \right)^0 - \left( \frac{1 - \sqrt{5}}{2} \right)^0 \right) = 0\\
        F_1 &= \frac{1}{\sqrt{5}} \left( \left( \frac{1 + \sqrt{5}}{2} \right)^1 - \left( \frac{1 - \sqrt{5}}{2} \right)^1 \right) = 1,
    \end{align*}
    so the formula is true for $n=0,1$. Now, we assume that it's true for $n$, and 
    so we need to show that it satisfies $F_{n+2} = F_n + F_{n+1}$. First, we add 
    $F_n$ and $F_{n+1}$:
    \begin{align*}
        F_n + F_{n+1}
        &= \frac{1}{\sqrt{5}}\Bigg(
        \left(\frac{1+\sqrt{5}}{2}\right)^n - \left(\frac{1-\sqrt{5}}{2}\right)^n
        + \left(\frac{1+\sqrt{5}}{2}\right)^{n+1} - \left(\frac{1-\sqrt{5}}{2}\right)^{n+1}
        \Bigg)\\
        &= \frac{1}{\sqrt{5}}\Bigg(
        \left(\frac{1+\sqrt{5}}{2}\right)^n\!\Big(1+\frac{1+\sqrt{5}}{2}\Big)
        - \left(\frac{1-\sqrt{5}}{2}\right)^n\!\Big(1+\frac{1-\sqrt{5}}{2}\Big)
        \Bigg)\\
    \end{align*}
    But notice that $1 + \frac{1 \pm \sqrt{5}}{2}
    = \frac{2 + 1 \pm \sqrt{5}}{2}
    = \frac{3 \pm \sqrt{5}}{2}
    = \left( \frac{1 \pm \sqrt{5}}{2} \right)^2$, so 
    we can simplify the remaining terms to:
    \begin{align*}
        &= \frac{1}{\sqrt{5}}\Bigg(
        \left(\frac{1+\sqrt{5}}{2}\right)^{n+2}
        - \left(\frac{1-\sqrt{5}}{2}\right)^{n+2}
        \Bigg)\\
        &= F_{n+2},
    \end{align*}
    which proves the formula.
    % Q1
\item We prove by strong induction on $n \ge 1$ that every $n$ can be written as
\[
n = a_1\cdot 1! + a_2\cdot 2! + a_3\cdot 3! + \cdots + a_k\cdot k!, \qquad 0 \le a_i \le i.
\]

Base case $n=1$: we have $1 = 1\cdot 1!$, so the statement is true for $n=1$.

For the induction step, assume the statement is true for all positive integers smaller than $n$.
Let $k$ be the largest integer such that $k! \le n$.  
Now divide $n$ by $k!$:
\[
n = q \cdot k! + r, \qquad 0 \le r < k!.
\]
Since $(k+1)! = (k+1)\cdot k! > n$, we know $q \le k$.  
We can then set $a_k = q$.  

The remainder $r$ is smaller than $n$, so by the induction hypothesis we can write
\[
r = a_1\cdot 1! + a_2\cdot 2! + \cdots + a_{k-1}\cdot (k-1)!,
\]
where each $a_i$ satisfies $0 \le a_i \le i$.  

Adding this to $a_k \cdot k!$ gives
\[
n = a_1\cdot 1! + a_2\cdot 2! + \cdots + a_k\cdot k!,
\]
with $0 \le a_i \le i$ for all $i$.  

So, by induction, every positive integer $n$ can be written in this form.

% Q2
\item For each $n\ge 0$, let $T_n$ be the number whose decimal form has $3^n$ ones.
We show by induction on $n$ that $3^n$ divides $T_n$.
A number with $3^n$ equal digits $d$ is $d\cdot T_n$, so this will prove the claim for any digit $d$.

Base cases: $T_0=1$ is divisible by $1$. Also $T_1=111=3\cdot 37$ is divisible by $3$.

Inductive step: suppose $3^n$ divides $T_n$.
Let $m=3^n$. The decimal for $T_{n+1}$ is three blocks of $m$ ones in a row, so
\[
T_{n+1}=T_n+10^{m}T_n+10^{2m}T_n=T_n\bigl(1+10^{m}+10^{2m}\bigr).
\]
When dividing by $3$, the remainder of $10$ is $1$, so the remainder of $10^{m}$ is also $1$.
Therefore $1+10^{m}+10^{2m}$ leaves remainder $1+1+1=3$, so it is divisible by $3$.
Since $3^n$ divides $T_n$ and $3$ divides $\bigl(1+10^{m}+10^{2m}\bigr)$, we get
$3^{n+1}$ divides $T_{n+1}$. 
Therefore by induction, any number with $3^n$ equal digits is divisible by $3^n$.

% Q3
\item We prove by induction that for every $n\ge 1$ there is an $n$-digit number,
all of whose digits are odd, that is divisible by $5^n$.

Base case $n=1$: the number $5$ works.

Inductive step: assume there is a $k$-digit number $N_k$ with all digits odd and $5^k$ divides $N_k$.
Write $N_k=5^k r$ for some integer $r$. We will choose an odd digit $t\in\{1,3,5,7,9\}$
so that, when dividing $2^k t + r$ by $5$, the remainder is $0$.

Note that when dividing by $5$, the possible remainders are $0,1,2,3,4$.
The digits $1,3,5,7,9$ leave remainders $1,3,0,2,4$ respectively, which are all different.
Multiplying each by $2^k$ and then dividing by $5$ again gives five remainders; they are still all different
(because if two of them were equal, then $5$ would divide $2^k(t_1-t_2)$, which would have to make $5$ dividing into $t_1-t_2$,
impossible for distinct $t_1,t_2\in\{1,3,5,7,9\}$). Hence these five values cover all five possible remainders.
Therefore we can choose $t$ so that $2^k t$ and $-r$ have the same remainder upon division by $5$,
which makes $2^k t + r$ divisible by $5$.

Now set
\[
N_{k+1}=t\cdot 10^k+N_k=t\cdot 2^k\cdot 5^k+5^k r=5^k(2^k t+r).
\]
So, $5$ divides $(2^k t+r)$, so $5^{k+1}$ divides $N_{k+1}$.
Also, $N_{k+1}$ has $k+1$ digits and all are odd. Hence proven.

% Q4
\item First we will show that it's true for the base cases, $n =1,2$.\\
Base cases: 

Case $n = 1$: we have $1 = F_2$, which is already a single Fibonacci number, so it works.

Case $n = 2$ we can write $2 = F_3$, and since there is only one term, there are no consecutive Fibonacci numbers used. Thus, the statement is true for $n = 1$ and $n = 2$.

Inductive step: assume the claim is true for all positive integers less than $n$.
Let $F_k$ be the largest Fibonacci number with $F_k\le n$.
If $n=F_k$, we are done.
Otherwise write $n=F_k+m$ with $m=n-F_k$.
Since $n<F_{k+1}=F_k+F_{k-1}$, we have $0<m<F_{k-1}$.
By the induction hypothesis, $m$ can be written as a sum of Fibonacci numbers,
all less than $F_{k-1}$, so none of them is $F_{k-1}$.
Also, because $F_{k-2}$ is next to $F_k$ in the list, $m$ cannot use $F_{k-2}$ either.
Therefore none of the terms used for $m$ is a neighbor of $F_k$,
and so $n=F_k+m$ is a sum with no two consecutive terms by induction.


% Q5
\item 
\[
\begin{aligned}
A(0,n)&=n+1.\\[4pt]
A(1,0)&=A(0,1)=2,\qquad A(1,n+1)=A(0,A(1,n))=A(1,n)+1\\
&\Rightarrow\ A(1,n)=n+2.\\[4pt]
A(2,0)&=A(1,1)=3,\qquad A(2, n+1)=A(1, A(2,n))=A(2,n)+2\\
&\Rightarrow\ A(2,n)=2n+3.\\[4pt]
A(3,0)&=A(2,1)=5,\qquad A(3,n+1)=A(2,A(3,n))=2A(3,n)+3.
\end{aligned}
\]
Solving $T_{n+1}=2T_n+3$ with $T_0=5$ gives $A(3,n)=2^{\,n+3}-3$.

For $m=4$,
\[
A(4,0)=A(3,1)=13,\qquad A(4,n+1)=A\bigl(3,\,A(4,n)\bigr)=2^{\,A(4,n)+3}-3.
\]
Thus
\[
A(4,1)=2^{\,16}-3,\qquad
A(4,2)=2^{\,2^{\,16}-3}-3,\qquad
A(4,3)=2^{\,2^{\,2^{\,16}}}-3,\ \ \text{and so on.}
\]
So, $A(4,n)$ is a tower of $n+3$ twos with a $-3$ at the end.

For $m=5$ we have
\[
A(5,0)=A(4,1)=2^{\,16}-3,\qquad A(5,n+1)=A\bigl(4,\,A(5,n)\bigr),
\]
so each step takes the previous value $x=A(5,n)$ and replaces it by
\[
A(5,n+1)=2^{\,2^{\,\cdot^{\,\cdot^{\,2^{\,x+3}}}}}-3,
\]
which is, a power tower of twos whose height is $x+3$, then subtract $3$.


% Q6
\item I couldn't really make much progress with this problem.

\begin{enumerate}
\item Pick an edge $e$ joining vertices $u$ and $v$.
If we remove $e$, we get a new graph $G\setminus e$.  
If we merge the two vertices together, we get another graph $G/e$.  
I think the idea is that we somehow compare how many colorings there are in each case...  
Maybe something like $C_G(n)$ equals one of them minus the other?  
I’m not totally sure why though, since I don’t fully see how the colorings correspond.

\item 
\item If there are no edges, then we can just pick any color for each vertex, so that part is easy.  
But once there are edges, I think we’re supposed to use the idea from the first part somehow.  
Maybe it has to do with removing an edge or something like that.  
I’m not exactly sure how to set that up though, and I got stuck trying to see how that would show it’s a polynomial.

\item  
If there are no edges, $C_G(n)=n^v$, so that part is easy.  
When there are edges, I think we again use induction, but I don’t know how to show that removing or merging edges keeps the degree the same.  
I got stuck trying to see why the degree would always be $v$.

\item 
We’re supposed to write 
\[
C_G(n)=a_k n^k+a_{k-1}n^{k-1}+\cdots+a_0,
\]
but I’m not sure how to show the signs change $+,-,+,-,$ and so on.  
Maybe it has to do with the subtraction in the formula from part (a),  
but I don’t know how to make that argument work. I got stuck here.
\end{enumerate}


\end{enumerate}
\end{document}